\documentclass[a4paper,10pt]{article}
\usepackage[utf8]{inputenc}
\usepackage[french]{babel}
\usepackage{amsfonts,bm,amsmath,amssymb,graphicx,verbatim}
\usepackage{cancel,url}

\newcommand{\dd}{\mathrm{d}}
\newcommand{\hf}{\hat{f}}
\newcommand{\real}{\mathbb{R}}
\newcommand{\integer}{\mathbb{Z}}
\newcommand{\definition}{\textbf{Definition }}
\newcommand{\exemples}{\textbf{Exemples }}
\newcommand{\remarque}{\textbf{Remarque }}
\newcommand{\proprietes}{\textbf{Propriétés }}
\newcommand{\propriete}{\textbf{Propriété }}

\title{Travaux pratiques: Génération de nombres aléatoires}
% \author{Orestis Malaspinas}
\date{A rendre pour le 01.06.2016}

\begin{document}
\maketitle

\section*{Travaux pratiques: Génération de nombres aléatoires}

Les nombres aléatoires sont utilisés dans un vaste champs d'applications (sécurité informatique, simulations, 
prise de décision, ...). Le but de ce TP est de faire un premier pas vers la compréhension de la problématique de 
la génération de nombres pseudo-aléatoires avec un ordinateur. 

\section*{Générateurs congruentiels linéaires}
Les générateurs congruentiels sont des générateurs assez simples et donnent des nombres aléatoires d'assez bonne qualité. 
La forme générale de ces générateurs est
\begin{equation}
 X_{i+1}=(aX_i+c)\mod m,
\end{equation}
où $a$, $c$, $m$ sont les paramètres dont la qualité des nombres aléatoires dépendront et $X_0$ est appelé le germe.

\begin{enumerate}
\item Cette congruence est de la forme $X_{i+1} = f(X_i)$. Si $0\leq X_i < n$ et $0\leq f (X_i ) < n$ $\forall i$, expliquez
pourquoi cette suite va commencer à faire des cycles (en d'autres termes $\exists m \geq 0$ tel que $X_{m+\lambda} = X_{\lambda}$).
 \item Créez une fonction pour générer $n$ nombres pseudo-aléatoires en appliquant une congruence linéaire avec paramètres $a$, $c$, $m$ et $X_0$.
 \item Écrivez un programme qui calcule la période d’un générateur congruentiel pour un ensemble de
paramètres données. Testez-le avec $m = 10$ et essayez de trouver un $a$ et un $c$ qui vous donnent
un cycle maximal.
\item Historiquement, le générateur \texttt{RANDU} développé par \texttt{IBM} utilisait les paramètres suivants :
$a = 65539$, $c = 0$, $n = 2^{31}$ and $X_0 = 123456789$. Générez 12000 nombres aléatoires avec
cette méthode. Visualisez les résultats comme des points $(X_{i-1} , X_i)$ en deux dimensions et
$(X_{i-2}, X_{i-1} , X_i )$ dans l’espace. En Matlab/Octave, servez-vous de la fonction plot3 pour afficher les triplets.
\end{enumerate}

\section*{Générateur de Stoll--Kirckpatrick}
Le générateur de Stoll-Kirckpatrick considère que nous possédons déjà 250 nombres aléatoires de
bonne qualité codés sur $l$ bits. Ensuite, nous pouvons trouver les autres nombres de la suite avec la
formule suivante :
\begin{equation}
X_{i+1} = X_{i-249}\ \bm{xor}\ X_{i-102},
\end{equation}
où $\bm{xor}$ est appliqué bit par bit dans la représentation binaire des nombres.
Prenez le fichier \texttt{aleatoire250.txt} disponible sur Cyberlearn. Ce fichier contient 250 nombres aléatoires (entre 0
et $2^{16}-1$) obtenus du site \url{http://www.random.org}. Utilisez l’algorithme ci-dessus pour en trouver 300000
nombres pseudo-aléatoires.
\begin{enumerate}
 \item Testez les résultats en les affichant en deux et trois dimensions comme avant.
 \item Calculez la moyenne des nombres obtenus. Comment se compare-t-elle à la moyenne théorique attendue?
\end{enumerate}


\section*{Calcul de $\pi$}

Utiliser les générateurs de nombres aléatoires précédemment implantés pour calculer $\pi$, ainsi que la fonction \texttt{rand} d'Octave/Matlab (ou
une librairie d'un langage utilisant l'algorithme Mersenne Twister, elle est disponible dans boost par exemple). Pour ce faire
il faut tirer des couples des points dont la position est aléatoire $(x,y)\in [0,1]\times[0,1]$. Il faut ensuite calculer le rapport du nombre de points
tirés qui satisfont $x^2+y^2\leq 1$ sur le nombre total de points tirés. Que doit donner ce rapport si les nombres tirés sont
tous équiprobables après un très grand nombre de tirages? Obtenez le plus grand nombre possible de décimales de $\pi$
avec les différents générateurs de nombres aléatoires. Qu'observez-vous?


\section*{Remarques}

Le travail peut-être effectué en groupe de deux, mais les rapports doivent être individuels (le code peut être identique, n'oubliez pas de mentionner 
explicitement si vous avez effectué le code à deux). 
Finalement, je dois pouvoir exécuter le code
afin de pouvoir reproduire les résultats présentés dans le rapport. 
Déposez le rapport EN FORMAT PDF et une archive contenant le code (deux fichiers séparés) 
sur le site du cours s'il vous plaît, cela simplifie mon administration (et évite les problèmes avec les étudiants qui 
ne mettent pas de nom sur le rapport...).

La note sera une combinaison entre le code rendu et le rapport (moitié/moitié). 



\end{document}
